% ------------------------------------------------------------------
% Aqui, o objetivo é mostrar como colocar uma tabela. Recomenda-se,
% quando necessário, a consulta no seguinte site:
%
%                https://www.tablesgenerator.com/
%
% Em que é possível criar suas próprias tabelas na mão, ou mesmo 
% importar do excel, por exemplo..
%-------------------------------------------------------------------

\chapter{tabelas}
\label{cap:tabelas}

%-------------------------------------------------------------------
% O que faremos abaixo é incluir uma tabela num ambiente table:
%-------------------------------------------------------------------

Criaremos uma tabela simples utilizando o ambiente tabular. Com o ambiente center, iremos centralizar:



\begin{table}[h!]
	\centering
	\caption{Exemplo de Tabela}
	\begin{tabular}{l c c c} 
	    \hline
		Linha 	& Valor 1 	& Valor 2 	& Valor 3 \\ \hline 
		1 		& 6 		& 87837 	& 787 \\ 
		2		& 7			& 78 		& 5415 \\
		3 		& 545 		& 778		& 7507 \\
		4 		& 545 		& 18744 	& 7560 \\
		5 		& 88 		& 788 		& 6344 \\
		\hline
	\end{tabular}
	\label{table:1} % Aqui 'nomeamos' a tabela
\end{table}

Obviamente, podemos criar tabelas maiores e mais robustas. Nosso objetivo aqui, entretanto, não é esse. Que fique claro que outras \textit{N} possibilidades estão disponíveis. Segue um outro exemplo, feito diretamente utilizando a linguagem R:

\begin{table}[!htbp] \centering 
	\caption{Ajustes nos parâmetros} 
	\label{} 
	\begin{tabular}{@{\extracolsep{5pt}}lccc} 
		\\[-1.8ex]\hline 
		\hline \\[-1.8ex] 
								& \multicolumn{3}{c}{\textit{Dependent variable:}} \\ 
		\cline{2-4} 
		\\[-1.8ex] 				& \multicolumn{2}{c}{Overall Rating} 	& High Rating \\ 
		\\[-1.8ex] 				& \multicolumn{2}{c}{\textit{OLS}} 		& \textit{probit} \\ 
		\\[-1.8ex]				& (1) 			& (2) 					& (3)\\ 
		\hline \\[-1.8ex] 
		Handling of Complaints 	& 0.692$^{***}$ & 0.682$^{***}$ 		&  \\ 
								& (0.149) 		& (0.129) 				&  \\ 
		No Special Privileges 	& $-$0.104 		& $-$0.103 				&  \\ 
								& (0.135) 		& (0.129) 				&  \\ 
		Opportunity to Learn 	& 0.249 		& 0.238$^{*}$ 			& 0.164$^{***}$ \\ 
								& (0.160) 		& (0.139) 				& (0.053) \\ 
		Performance-Based Raises& $-$0.033 		& 						&  \\ 
								& (0.202) 		&  						&  \\ 
		Too Critical			& 0.015 		&  						& $-$0.001 \\ 
								& (0.147)		&  						& (0.044) \\ 
		Advancement 			&  				&  						& $-$0.062 \\ 
								&  				&  						& (0.042) \\ 
		Constant 				& 11.011 		& 11.258 				& $-$7.476$^{**}$ \\ 
								& (11.704) 		& (7.318) 				& (3.570) \\ 
		\hline \\[-1.8ex] 
		Observations 			& 30 			& 30 					& 30 \\ 
		R$^{2}$ 				& 0.715			& 0.715 				&  \\ 
		Adjusted R$^{2}$ 		& 0.656 		& 0.682 				&  \\ 
		Akaike Inf. Crit. 		& 				&  						& 26.175 \\ 
		\hline 
		\hline \\[-1.8ex] 
		\textit{Note:}  		& \multicolumn{3}{r}{$^{*}$p$<$0.1; $^{**}$p$<$0.05; $^{***}$p$<$0.01} \\ 
	\end{tabular} 
	\label{table:2}
\end{table} 


%-------------------------------------------------------------------
% mais lipsum...
%-------------------------------------------------------------------

\lipsum[2-4]
